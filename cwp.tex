\documentclass[a4paper]{article}
\usepackage{cwpuzzle}
\usepackage{multicol}
\usepackage{german}
\setlength{\textwidth}{13cm}
\begin{document}

\begin{Puzzle}{6}{7}%
|[1]c|[2]o|[3]u|n|[4]t|[5]e|[6]r|.
|[7]h|e|x|*|[8]r|s|a|.
|[9]a|m|t|*|[10]a|n|d|.
|r|*|[11]e|[12]s|c|*|*|.
|*|[13]f|r|i|e|[14]n|[15]d|.
|[16]v|i|m|*|*|[17]o|r|.
\end{Puzzle}

\begin{multicols}{2}%
\begin{PuzzleClues}{\textbf{Horizontal:}}\\%
	\Clue{1}{counter}{Something increasing each time around}%
	\Clue{7}{hex}{Regular polygon with the most vertices to tile a plane with}%
	\Clue{8}{rsa}{First algorithm known to be suitable for signing as well as encryption}%
	\Clue{9}{amt}{Intels hardware-based remote management}%
	\Clue{10}{and}{Truth table: $0001$}%
	\Clue{11}{esc}{Probably the leftest, most upperest key on your keyboard}%
	\Clue{13}{friend}{Outside of a dog it is a book. Inside of a dog it is too dark to read. (\textit{Groucho Marx})}%
	\Clue{16}{vim}{Not EMACS}%
	\Clue{17}{or}{Truth table: $0111$}%
\end{PuzzleClues}%

\begin{PuzzleClues}{\textbf{Vertical:}}\\%
	\Clue{1}{char}{Data type holding one byte}%
	\Clue{2}{oem}{Original equipment manufacturer}%
	\Clue{3}{uxterm}{X terminal emulator for Unicode environments}%
	\Clue{4}{trace}{Following a packet through the internet}%
	\Clue{5}{esn}{A recurrent neural network with a sparsely connected hidden layer}%
	\Clue{6}{rad}{An iterative application development methodology}%
	\Clue{12}{si}{International System of Units}%
	\Clue{13}{fi}{ISO country code of Mr. Torvalds Homecountry}%
	\Clue{14}{no}{Connected to \textbf{\textit{Clue 15}} the first \textit{Bond} movie villain}%
	\Clue{15}{dr}{Without \textbf{\textit{Clue 14}} rather respectable}%
\end{PuzzleClues}%
\end{multicols}%

\end{document}
